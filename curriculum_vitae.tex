\documentclass[11pt]{article}
\nocite{*}
\usepackage{newtxtext,newtxmath}

\usepackage{graphics}
\usepackage{tcolorbox}
\usepackage{color}
\usepackage[margin=0.5in]{geometry}
\usepackage[hyphens]{url}
\usepackage{amsmath}
\usepackage{amsfonts}
\usepackage{comment}
\usepackage{pdfpages}
\usepackage{titlesec}
\usepackage{enumitem}
\usepackage[font=small]{caption,subcaption}
\usepackage{graphicx}
\usepackage{wrapfig}
\usepackage[normalem]{ulem}
\usepackage{titling}
\usepackage{hyperref}
\usepackage{floatrow}
\setlength{\droptitle}{-8em}

\titleformat{\section}{\normalfont\fontsize{14}{15}\bfseries\itshape}{\thesection}{1em}{}
\titlespacing*{\section}{0pt}{0.2in}{0.1in}
\pagenumbering{gobble}

\title{Curriculum Vitae of Matthew Pisano}
\author{\textbf{\LARGE Matthew Thomas Pisano}} 
\date{}

\begin{document}
\pagestyle{plain}

\begin{tcolorbox}[width=\linewidth, sharp corners=all, colback=white!90!green, colframe=black]
    
    \begin{center}
        Curriculum Vitae\\
        \vspace{0.1in}
        \textbf{\LARGE Matthew Thomas Pisano}\\
        \vspace{0.1in}
        \includegraphics[width=11pt,trim=0 0.4in 0 0]{images/phone.png}
        (845)-706-0677 
        \hspace*{0.3in}
        \includegraphics[width=11pt,trim=0 0.4in 0 0]{images/mail.png}
        \href{mailto:matthewpisano14@gmail.com}{matthewpisano14@gmail.com} 
        \hspace*{0.5in}
        \includegraphics[width=11pt,trim=0 0.4in 0 0]{images/linkedin.png}
        \href{https://www.linkedin.com/in/matthew-pisano/}{matthew-pisano}

        \vspace{0.1in}
        \includegraphics[width=11pt,trim=0 0.4in 0 0]{images/github.png}
        \href{https://github.com/matthew-pisano}{matthew-pisano}
        \hspace*{0.5in}
        \includegraphics[width=11pt,trim=0 0.4in 0 0]{images/website.png}
        \href{https://matthewpisano.com/}{matthewpisano.com}
    \end{center}
    
\end{tcolorbox}

\vspace{0.1in}

\section*{Professional Summary}

Artificial intelligence research and software engineer with interests centered on AI alignment, mechanistic interpretability, and online learning. Graduate of Rensselaer Polytechnic Institute with a Master's of Science in Computer Science. Research experience in industry and academic AI labs, published research, and lecture experience. Proven ability to create novel and rigorous research, efficiently design technical software, and lead groups of both researchers and professional software engineers. Additional experience teaching courses in computer science as both an adjunct instructor and teaching assistant.

\section*{Skills and Experience}
\begin{itemize}
    
    \item \textbf{Research} \textit{(3 years)}: Peer-reviewed publications, academic talk invitations, work in industry/academic research labs.
    \item \textbf{Teaching} \textit{(3 years)}: Experience in independently holding lectures, managing curricula, testing and grading.
    \item \textbf{Machine Learning} \textit{(3 years)}: Transformer models, AI alignment research, RL, NLP, ASR, PyTorch, CUDA.
    \item \textbf{Python Programming} \textit{(5 years)}: PyTorch, HuggingFace, data analysis, LLM fine-tuning, distributed systems.
    \item \textbf{Software Development} \textit{(5 years)}: DevOps, CI/CD, Agile methodologies, web development, project leadership.
    \item \textbf{Awards and Certifications:} SUNY New Paltz outstanding graduate, Phi Theta Kappa Honor Society, $1^{st}$ place globally at Mega-Ace hackathon, Eagle Scout.

\end{itemize}

\section*{Research and Publications}
\begin{itemize}
    
    \item \textbf{Bergeron: Combating Adversarial Attacks through a Conscience-Based Alignment Framework}.  A weak-to-strong generalization framework for alignment.  Involves an LLM acting as the ``conscience'' of a larger, more capable LLM.  Accepted at the RPI Graduate Research Symposium.  Published to \href{https://www.proquest.com/openview/fc38d3daf9e6be8598ba7ec38aa7d3af/1?pq-origsite=gscholar&cbl=18750&diss=y}{\textit{ProQuest}}.  \href{https://arxiv.org/abs/2312.00029}{ArXiv: 2312.00029}
    
    \item \textbf{Moral High Ground: A Text-Based Games Benchmark for Moral Evaluation}, under \textit{IBM Research}.  A novel benchmark for evaluating the moral reasoning abilities of LLMs through conversational text-based games.
    
    \item \textbf{PredictChain: Empowering Collaboration and Data Accessibility for AI in an Algorand Blockchain-based Marketplace}.  Research paper on the development of \textit{PredictChain}, a decentralized machine learning marketplace.  $1^{st}$ place global hackathon winner, additional award for \textit{Most Innovative Use of Technology}.  Presented at \textit{ChainScience} 2023. \href{https://arxiv.org/abs/2307.15168}{ArXiv: 2307.15168}
    
    \item \textbf{On Picard Groups and Jacobians of Directed Graphs}. Linear algebra and combinatorics study of \textit{Chip-Firing games} and how graph edge manipulations affect game state evolution.  Presented at \textit{JMM} 2023 and published in the journal \href{https://doi.org/10.1016/j.laa.2025.02.020}{\textit{Linear Algebra and its Applications} Vol. 711, P. 180-211}.
    
\end{itemize}

\section*{Education}
\begin{itemize}

    \item \textbf{Rensselaer Polytechnic Institute} (2023-2024), Troy, NY. \textit{Master of Science} in Computer Science, published thesis on artificial intelligence alignment.  4.0 GPA, awarded TA position and scholarship.  Classes in cognitive science, informatics, learning theory, reinforcement learning, information retrieval, parallel computing, and program analysis.
    
    \item \textbf{SUNY New Paltz} (2021-2022), New Paltz, NY. \textit{Bachelor of Science} in Computer Science (Minor in Applied Mathematics), undergraduate research. 4.0 GPA, \textit{Outstanding Graduate honor}, and published undergraduate research.
    
    \item \textbf{SUNY Ulster Community College} (2019-2021), Stone Ridge, NY. \textit{Associate of Science} in Computer Science (Certifications in Web and Mobile app development). 3.98 GPA, \textit{Phi Theta Kappa Honor Society}, tutoring in mathematics.

\end{itemize}

\section*{Work Experience}
\begin{itemize}
    
    \item \textbf{IBM}: \textit{Staff Software Engineer} (2025-Present), Poughkeepsie, NY. Low-level C/C++ development for compiling PyTorch models onto IBM's \textit{Sypre} AI accelerator chips.  Work with attention mechanism optimization, tensor manipulation, and PyTorch graphs. Participation in regular volunteering, community outreach, and education programs.
    
    \item \textbf{FileScience}: \textit{Quality Assurance Lead Engineer} (2020-2025), Kingston, NY. Develop complex, distributed cloud-to-cloud backup software and ensure code is up to industry standard testing and documentation practices.  Coordinate with team members to regularly review algorithms and methodologies.

\end{itemize}

\section*{Research Experience}
\begin{itemize}

    \item \textbf{Substrate AI}: \textit{Research Engineer} (2024), Madrid, Spain. Development of a small \textit{Metacontrol} LLM that evaluates whether user queries violate a policy set before being sent to a more generalized assistant.  Used \textit{LoRA} fine-tuning and dataset cleaning to generate high-quality results from sparse, synthetically generated data.

    \item \textbf{IBM Research}: \textit{Research Extern} (2023), Yorktown Heights, NY. Research into LLM alignment using moral principles through fine-tuning on text-based games. Generated a diverse conversational dataset of moral situations and trained LLMs to extract the embedded ethical principals.
    
    \item \textbf{SUNY New Paltz}: \textit{Research Assistant} (2022), New Paltz, NY. Research into chip firing games and programmatically manipulating directed graphs. Mentored by Jaiung Jun under the department of mathematics.

\end{itemize}

\section*{Teaching Experience}
\begin{itemize}

    \item \textbf{SUNY Ulster}: \textit{Computer Science Adjunct} (2025-Present), Stone Ridge, NY. Instruct \textit{Computer Architecture and Organization} course.  Independently lecture, manage curriculum, grade student work, and coordinate with colleagues.
    
    \item \textbf{Rensselaer Polytechnic Institute}: \textit{Graduate Teaching Assistant} (2023-2024), Troy, NY. Provide students with assistance, grade assignments, and develop exam questions for the \textit{Principles of Software} and \textit{Data Structures} classes.

\end{itemize}

\section*{Presentations and Lectures}

See: \href{https://matthewpisano.com/lectures}{\textit{matthewpisano.com/lectures}}

\begin{itemize}

    \item \textbf{Transformer Models: Architectures and Use Cases}: A technical lecture on the transformer architecture, common model implementations, and the attention mechanism. Given as a part of IBM's internal technical lecture series.

    \item \textbf{Computer Architecture and Organization Lecture Series}: Lecture series associated with SUNY Ulster's \textit{Computer Architecture and Organization} course. 28 individual lectures covering MIPS assembly, binary arithmetic, computer architecture, virtualization, cloud computing, and low-level vulnerabilities.

    \item \textbf{Bergeron: Combating Adversarial Attacks by Emulating a Conscience}: Presentation associated with Bergeron research and Master's thesis. Presented as two guest lectures for RPI cognitive science students, as thesis defense, and at RPI's Graduate Research Symposium.
    
    \item \textbf{PredictChain: Empowering Collaboration for AI in a Blockchain-based Marketplace}: Presentation for PredictChain research project. Presented at the 2023 Mega Ace Hackathon where it won first place globally and at the \textit{ChainScience} 2023 conference.
    
    \item \textbf{Homophone Decoding and Speech Based Emotion Detection}: Presentation for a study, commissioned as a part of an RPI-IBM collaboration, focused on developing an automatic speech recognition (ASR) model with better accuracy on homophones. Presented to principal investigators and IBM grantors.
    
    \item \textbf{On Picard Groups and Jacobians of Directed Graphs}: Presentation for chip firing research. Presented at \textit{Joint Mathematics Meetings} 2023 and as a guest lecture for the mathematics department at SUNY New Paltz.

\end{itemize}

\section*{Academic Service}
\begin{itemize}

    \item \textbf{European Journal of Computer Sciences and Informatics}: Participate in the peer-review process as a reviewer, analyzing submissions and giving feedback to authors.
    
    \item \textbf{IBM Research Internship Symposium}: Participate in the peer-review process as a reviewer, give feedback to interns on their project and associated paper.
    
    \item \textbf{Marist Computing Conference}: Serve as a judge of student poster presentations and research projects, nominate projects for conference awards.

\end{itemize}

\section*{Projects}
\begin{itemize}

    \item \textbf{Bergeron}: Research project to investigate the abilities of a smaller ``conscience'' language model to protect a much larger language model against adversarial attacks. Protected models resistant to up to 97\% of attacks. Developed as part of Master's thesis. \href{https://matthewpisano.com/research/bergeron}{\textit{matthewpisano.com/research/bergeron}}.

    \item \textbf{MASM}: A fully-featured MIPS assembler, interpreter, and debugger written from scratch in C++. Developed as a learning aide to SUNY Ulster's \textit{Computer Architecture and Organization} course. \href{https://matthewpisano.com/personal/masm}{\textit{matthewpisano.com/personal/masm}}.
    
    \item \textbf{Manifest Destiny}: Massively parallel population simulation written in C and CUDA. Designed to run across multiple GPUs and nodes using MPI and Slurm for coordination. Developed for RPI's class on parallel computing along with a corresponding research report. \href{https://matthewpisano.com/school/destiny}{\textit{matthewpisano.com/school/destiny}}.

    \item \textbf{PredictChain}: A decentralized marketplace for predictive AI models. Users are able to upload datasets, request the training of predictive models, or submit queries to those models. \href{https://matthewpisano.com/research/predictChain}{\textit{matthewpisano.com/research/predictChain}}.

    \item \textbf{Chip Firing Simulation}: A detailed chip firing game simulator designed to record game behavior and extract patterns from a graph's node and edge direction permutations. \href{https://matthewpisano.com/research/chipFiring}{\textit{matthewpisano.com/research/chipFiring}}.

\end{itemize}

\section*{Volunteer Work}
\begin{itemize}

    \item \textbf{SUNY New Paltz Capstone Judge}: Judge undergraduate student work on capstone projects and provide feedback based on completion, quality, and approach.

    \item \textbf{Workshops and Career Fairs}: Work with local college students to improve job-market skills, like résumé building. Participate in career fairs to connect students with potential opportunities at IBM.

    \item \textbf{Engineers Week}: Annual event sponsored by IBM where engineers visit primary and secondary schools to speak with students, educate them on the field of engineering, and on how to become future engineers. Visits involve lectures and hands-on activities with students.
    
    \item \textbf{Community Outreach Events}: Organized and participated in outreach events to schools in the Hudson Valley area for education on STEM and potential careers for students.

\end{itemize}

\end{document}